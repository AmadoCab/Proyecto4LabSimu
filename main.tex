\documentclass[
    english, spanish, Ce-table, Ce-theorem
]{CabesHW}
\usepackage[utf8]{inputenc}
\usepackage{parskip}
\usepackage[es-tabla]{babel}
\usepackage{hyperref}
\usepackage{algpseudocodex}
\usepackage{minted}
\usepackage{siunitx}

%%%%%%%%%%%%%%%%%%%%%%%%%%%%%%%%%
%% Configuraciones adicionales %%
%%%%%%%%%%%%%%%%%%%%%%%%%%%%%%%%%
\makeatletter

%% Algpseudocodex restyling %%
\renewcommand{\algpx@commentFormat}[1]{%
	\ifbool{algpx@italicComments}{%
		\textsl{\textcolor{\algpx@commentColor}{#1}}%
	}{%
		\textcolor{\algpx@commentColor}{#1}%
	}%
}

%% myalg environment %%
\newenvironment{myalg}{%algorithm
    \begin{minipage}{0.8\textwidth}
    \selectlanguage{english}
    \begin{algorithmic}%
}{%
    \end{algorithmic}
    \end{minipage}%
}

%% Float new environments %%
\newfloat{algorithm}{tbp}{loa}
\floatname{algorithm}{Algoritmo}

\makeatother
\endinput
\decimalpoint
\usemintedstyle{lovelace}
\hypersetup{
    colorlinks=true,
    allcolors=blue!65!black
}

\institute{Escuela de Ciencias Físicas y Matemática}
\title{Proyecto 4}
\author{Amado Cabrera, Denilson Mendoza y Mariana Pérez}
\id{201905757, 202005724 y 201901040}
\date{\today}
\course{Laboratorio de simulación}
\professor{Carlos Soto}

\begin{document}

\maketitle

\begin{abstract}
Este proyecto de clase consistió en aplicar y comparar tres métodos numéricos para resolver ecuaciones diferenciales ordinarias: Euler, Taylor y Runge--Kutta, siendo estos dos últimos de cuarto orden. El objetivo de este proyecto era analizar el rendimiento de estos métodos al comparar su precisión. Para cada método, se escribió un programa en C para calcular el valor de una ecuación de crecimiento poblacional. A continuación, los programas fueron probados y comparados. El método de Euler, que es un método de primer orden, resultó ser el menos preciso y eficiente de los tres métodos. El método de Taylor, que es un método de orden superior, era más preciso pero es más complejo de calcular. Por último, el método Runge--Kutta, que es un método de cuarto orden, resultó ser el más preciso y eficaz de los tres.
\end{abstract}

\vspace{0.5em}
\section{Marco teórico}
\subsection{Método de Euler}
El método de Euler permite obtener aproximaciones $w_i \approx y_{(x_i)}$ al problema de valor inicial bien planteado
\begin{align*}
\label{1.1}
    \td{y}{x} = f_{(x,y)}, \qquad a \leq x \leq b; \qquad y_{(a)} = \alpha. \tag{1.1}
\end{align*}

Este método no brinda una aproximación continua a la solución $y_{(x)}$, sino que genera aproximaciones de $y_{(x)}$ para varios valores de $x$ en el intervalo $[a, b]$. Dichos valores, denotados como $x_i$ y llamados \textit{puntos de malla}, deben estar distribuidos equitativamente en el intervalo. Para ello, se toma un número entero positivo $N$ (que representa la cantidad de pasos en el intervalo) y se seleccionan los puntos de malla
\[ x_i = a + ih, \qquad \text{para cada $i = 0, 1, \ldots, N$}; \]
donde $h = (b-a)/N = x_{i+1} - x_i$ es la distancia que existe entre cada $x_i$, por lo que $h$ recibe el nombre de \textit{tamaño del paso}.

Se usará el teorema de Taylor para derivar el método de Euler. Suponer que la solución de la ecuación  \eqref{1.1} tiene dos derivadas continuas en el intervalo $[a, b]$, por lo que para cada $i = 0, 1, \ldots, N-1$
\begin{align*}
     y_{(x)} &= y_{x_i} + (x-x_i)y'_{(x_i)} + \frac{(x-x_i)^2}{2!}y''_{(\xi_i)}.
\end{align*} 
Valuando la expresión en $x=x_{i+1}$:
\begin{align*}
     y_{(x_{i+1})} &= y_{x_i} + (x_{i+1}-x_i)y'_{(x_i)} + \frac{(x_{i+1}-x_i)^2}{2!}y''_{(\xi_i)}\\
     &= y_{x_i} + hy'_{(x_i)} + \frac{h^2}{2!}y''_{(\xi_i)}.
\end{align*}
El método de Euler se consigue descartando el término restante que incluye $\xi_i$:
\begin{align*}
    w_0 &= \alpha,\\
    w_{i+1} &= w_i + h f_{(t_i, w_i)}, \qquad \text{para cada $i = 0, 1, \ldots, N-1$}.
\end{align*}

\subsubsection{Pseudo--código}
\begin{algorithm}[H]
    \centering
    \begin{myalg}[1]
    \LComment{Método de Euler para resolución numérica de EDO}
    \Function{Edo}{$x$, $y$}
        \State \Output $0.16 y$
    \EndFunction
    \State \phantom{}
    \State $x_i \gets 0$ \Comment{Valor inicial de $x$}
    \State $y_i \gets 367$ \Comment{Valor inicial de $y$}
    \State $x_f \gets 20$ \Comment{Valor final de $x$}
    \State $n \gets 100$ \Comment{Número máximo de pasos}
    \State \phantom{}
    \State $h \gets (x_f - x_i)/n$ \Comment{Valor del paso}
    \For{$i = 0,\ldots,n$}
        \State \Output $(i, x_i, y_i)$
        \State \phantom{}
        \State $y_i \gets y_i + h\cdot\Call{Edo}{x_i, y_i}$
        \State $x_i \gets x_i + h$
    \EndFor
    \end{myalg}
    \caption{Pseudo--código para el método de Euler.}
    \label{alg:euler}
\end{algorithm}

\vspace{1em}
\subsection{Método de Taylor}
Suponer que la solución a la ecuación
\[ \td{y}{x} = f_{(x,y)}, \qquad a \leq x \leq b; \qquad y_{(a)} = \alpha \]
tiene $n+1$ derivadas continuas.

Derivando repetidas veces la solución se obtiene
\begin{align*}
\label{tay2.1}
    y^{(k)}_{(x)} &= f^{(k-1)}_{(x, y_{(x)})}. \tag{2.1}
\end{align*}

Empleando el $n$-ésimo polinomio de Taylor al rededor de $x_i$, expandir la solución $y_{(x)}$ y posteriormente valuarla en $x= x_{i+1}$:
\begin{align*}
    y_{(x)} &= y_{x_i} + (x-x_i)y'_{(x_i)} + \frac{(x-x_i)^2}{2!}y''_{(x_i)} + \ldots + \frac{(x-x_i)^{n}}{n!}y^{(n)}_{(x_i)} + \frac{(t-x_i)^{n+1}}{(n+1)!}y^{(n+1)}_{(\xi_i)}\\
    y_{(x_{i+1})} &= y_{x_i} + (x_{i+1}-x_i)y'_{(x_i)} + \frac{(x_{i+1}-x_i)^2}{2!}y''_{(x_i)} + \ldots + \frac{(x_{i+1}-x_i)^{n}}{n!}y^{(n)}_{(\xi_i)} + \frac{(x_{i+1}-x_i)^{n+1}}{(n+1)!}y^{(n+1)}_{(x_i)}\\
\label{tay2.2}
    y_{(x_{i+1})} &= y_{x_i} + hy'_{(x_i)} + \frac{h^2}{2!}y''_{(x_i)} + \ldots + \frac{h^{n}}{n!}y^{(n)}_{(x_i)} + \frac{h^{n+1}}{(n+1)!}y^{(n+1)}_{(\xi_i)}, \tag{2.2}
\end{align*}
para algún $\xi_i$ en $(x_i, x_{i+1})$.

Reemplazando \eqref{tay2.1} en \eqref{tay2.2}, se consigue
\begin{align*}
    y_{(x_{i+1})} &= y_{x_i} + hf_{(x_i, y_{(x_i)})} + \frac{h^2}{2!}f'_{(x_i, y_{(x_i)})} + \ldots + \frac{h^{n}}{n!}f^{(n-1)}_{(x_i, y_{(x_i)})} + \frac{h^{n+1}}{(n+1)!}f^{(n)}_{(\xi_i, y_{(\xi_i)})}
\end{align*}

El método de Taylor de orden $n$ se consigue descartando el término restante que incluye $\xi_i$:
\begin{align*}
    w_0 &= \alpha,\\
    w_{i+1} &= w_i + hT^{(n)}_{(x_i, w_i)}, \qquad \text{para cada $i = 0, 1, \ldots, N-1$};
\end{align*}
donde
\begin{align*}
    T^{(n)}_{(x_i, w_i)} &= f_{(x_i, y_{(x_i)})} + \frac{h}{2!}f'_{(x_i, y_{(x_i)})} + \ldots + \frac{h^{n-1}}{n!}f^{(n-1)}_{(x_i, y_{(x_i)})} .
\end{align*}

\subsubsection{Pseudo--código}
\begin{algorithm}[H]
    \centering
    \begin{myalg}[1]
    \LComment{Método de Taylor para resolución numérica de EDO}
    \Function{Edo}{$x$, $y$} \Comment{Edo y sus derivadas}
        \State \Output $0.16 y$
    \EndFunction
    \State \phantom{}
    \Function{dEdo}{$x$, $y$} \Comment{Derivada 1}
        \State \Output $0.16 \Call{Edo}{x,y}$
    \EndFunction
    \State \phantom{}
    \Function{ddEdo}{$x$, $y$} \Comment{Derivada 2}
        \State \Output $0.16 \Call{dEdo}{x,y}$
    \EndFunction
    \State \phantom{}
    \Function{dddEdo}{$x$, $y$} \Comment{Derivada 3}
        \State \Output $0.16 \Call{ddEdo}{x,y}$
    \EndFunction
    \State \phantom{}
    \State $x_i \gets 0$ \Comment{Valor inicial de $x$}
    \State $y_i \gets 367$ \Comment{Valor inicial de $y$}
    \State $x_f \gets 20$ \Comment{Valor final de $x$}
    \State $n \gets 100$ \Comment{Número máximo de pasos}
    \State \phantom{}
    \State $h \gets (x_f - x_i)/n$ \Comment{Valor del paso}
    \For{$i = 0,\ldots,n$}
        \State \Output $(i, x_i, y_i)$
        \State \phantom{}
        \State $y_i \gets y_i + h(\Call{Edo}{x_i, y_i} + h\cdot\Call{dEdo}{x_i, y_i}/2 + $
        \Statex \hspace{5em}$+ h^2\cdot\Call{ddEdo}{x_i, y_i}/6 + h^3\cdot\Call{dddEdo}{x_i, y_i}/24)$
        \State $x_i \gets x_i + h$
    \EndFor
    \end{myalg}
    \caption{Pseudo--código para el método de Taylor.}
    \label{alg:taylor}
\end{algorithm}

\vspace{1em}
\subsection{Método de Runge--Kutta}
Los métodos de Runge-Kutta tienen el error de truncamiento local de alto orden de los métodos de Taylor, pero eliminan la necesidad de calcular y evaluar las derivadas de $f_{(t, y)}$.

El método de Runge--Kutta de orden $n$ se deriva a partir de aproximar
\begin{align*}
    T^{(n)}_{(x_i, w_i)} &= f_{(x_i, y_{(x_i)})} + \frac{h}{2!}f'_{(x_i, y_{(x_i)})} + \ldots + \frac{h^{n-1}}{n!}f^{(n-1)}_{(x_i, y_{(x_i)})}. 
\end{align*}


El método de Runge--Kutta de cuarto orden es el método más utilizado y está dado por:
\begin{align*}
    w_0 &= \alpha,\\
    w_{i+1} &= w_i + \frac{1}{6}(k_1 + 2k_2 + 2k_3 + k_4), \qquad \text{para cada $i= 0, 1, \ldots, N-1$;}
\end{align*}
donde
\begin{align*}
    k_1 &= hf_{(x_i, w_i)}, \\
    k_2 &= hf_{(x_i + \frac{h}{2}, w_i + \frac{1}{2}k_1)},\\
    k_3 &= hf_{(x_i + \frac{h}{2}, w_i + \frac{1}{2}k_2)}, \\
    k_4 &= hf_{(x_{i+1}, w_i + k_3)}.
\end{align*}

\subsubsection{Pseudo--código}
\begin{algorithm}[H]
    \centering
    \begin{myalg}[1]
    \LComment{Método de Runge--Kutta para resolución numérica de EDO}
    \Function{Edo}{$x$, $y$}
        \State \Output $0.16 y$
    \EndFunction
    \State \phantom{}
    \State $x_i \gets 0$ \Comment{Valor inicial de $x$}
    \State $y_i \gets 367$ \Comment{Valor inicial de $y$}
    \State $x_f \gets 20$ \Comment{Valor final de $x$}
    \State $n \gets 100$ \Comment{Número máximo de pasos}
    \State \phantom{}
    \State $h \gets (x_f - x_i)/n$ \Comment{Valor del paso}
    \For{$i = 0,\ldots,n$}
        \State \Output $(i, x_i, y_i)$
        \State \phantom{}
        \State $k_1 \gets h\cdot\Call{Edo}{x_i, y_i}$
        \State $k_2 \gets h\cdot\Call{Edo}{x_i+h/2, y_i+k_1/2}$
        \State $k_3 \gets h\cdot\Call{Edo}{x_i+h/2, y_i+k_2/2}$
        \State $k_4 \gets h\cdot\Call{Edo}{x_i+h, y_i+k_3}$
        \State \phantom{}
        \State $y_i \gets y_i + h(k_1 + 2k_2 + 2k_3 + k_4)/6$
        \State $x_i \gets x_i + h$
    \EndFor
    \end{myalg}
    \caption{Pseudo--código para el método de Runge--Kutta.}
    \label{alg:runge-kutta}
\end{algorithm}

\section{Resultados}
Los métodos numéricos fueron todos implementados en el lenguaje de programación C, todos para resolver la ecuación diferencial ordinaria (EDO) de crecimiento poblacional,

\[ \td{y}{x} = 0.16y_{(x)}, \quad 0\leq x\leq 20;\] 
\[ y_{(0)} = 367. \]

Para automatizar el proceso de testeo y depuración de cada uno de los códigos se escribió un \textit{script} en Bash, que se encargaba de compilar, ejecutar y colocar los datos resultantes de los cálculos llevados a cabo por cada método.

Para los cálculos del error entre el valor real y los métodos numéricos se utilizó Mathematica, más concretamente un archivo \texttt{.m} (\textit{scripts} de Mathematica) para poder ejecutar el programa desde terminal y facilitar su automatización (aunque requiere el paquete \texttt{MaTeX} para ejecutarse correctamente). Aprovechando la facilidad de trabajar datos en Mathematica se utilizó también para visualizar datos en tablas y dar una buena presentación de los mismos.

Por último, todo lo anteriormente recopilado se utilizó para realizar el presente reporte en \LaTeX{}. Aquí se usaron algunos archivos auxiliares para estilizar el documento, configurar paquetes y componer las tablas por medio de los archivos \texttt{.csv}. Haciendo uso también de la herramienta de manejo de versiones Git, para que todo el equipo pudiera trabajar a la vez en el proyecto (\href{https://github.com/AmadoCab/Proyecto4LabSimu}{\sffamily enlace repositorio}).

(Cabe mencionar que en el repositorio todos los \textit{commits} fueron realizados desde una misma cuenta ya que las herramientas para trabajar en Git que se utilizaron lo permitieron de esa manera.)

\vspace{0.5em}
\subsection{Tabulación de datos obtenidos}
La última columna de las tablas a continuación muestra que el método de Euler es el que devuelve los valores más alejados del valor real de la solución de la EDO.

Ya que los resultados en las tablas fueron redondeados a 3 cifras significativas, parece que el error para los métodos de Taylor y Runge--Kutta es igual. Si se incrementa a 9 cifras significativas, es posible apreciar la diferencia.

\[ 
\begin{array}{r@{\ }l}
\text{Método de Taylor:}& \num{0.000 245 132 092},\\
\text{Método de Runge--Kutta:}& \num{0.000 245 132 091}.
\end{array}
\]

Con lo anterior puede verificarse que para esta EDO de crecimiento poblacional, el método que brinda las mejores aproximaciones es el de Runge--Kutta.

\begin{center}
\begin{minipage}{.45\textwidth}
\begin{table}[H]
    \centering
    \begin{tabular}{c|ccc}
    $i$ & $x_i$ & $y_i$ & $\varepsilon_i$\\[.1em]
    \hline\\[-.9em]
    \input{|python3 data/tabulardatos.py "data/data-eulerEr.csv"}
    \end{tabular}
    \caption{Datos obtenidos por el método de Euler.}
    \label{tab:euler}
\end{table}
\end{minipage}
\hfill
\begin{minipage}{.45\textwidth}
\begin{table}[H]
    \centering
    \begin{tabular}{c|ccc}
    $i$ & $x_i$ & $y_i$ & $\varepsilon_i$\\[.1em]
    \hline\\[-.9em]
    \input{|python3 data/tabulardatos.py "data/data-taylorEr.csv"}
    \end{tabular}
    \caption{Datos obtenidos por el método de Taylor.}
    \label{tab:taylor}
\end{table}
\end{minipage}
\end{center}

\vspace{1em}

\begin{table}[H]
    \centering
    \begin{tabular}{c|ccc}
    $i$ & $x_i$ & $y_i$ & $\varepsilon_i$\\[.1em]
    \hline\\[-.9em]
    \input{|python3 data/tabulardatos.py "data/data-rungekuttaEr.csv"}
    \end{tabular}
    \caption{Datos obtenidos por el método de Runge--Kutta.}
    \label{tab:runge-kutta}
\end{table}

\vspace{0.5em}
\subsection[Gráfica comparativa de las soluciones]{Gráfica comparativa de las soluciones $\boldsymbol{y_{(x)}}$}
En esta gráfica no es posible apreciar la curva para el error empleando el método de Taylor porque esta coincide en gran medida con la curva para el error empleando el método de Runge--Kutta.

\begin{figure}[H]
    \centering
    \includegraphics{imgs/plot-comp.pdf}
    \caption{Gráfica comparativa del error después de 100 iteraciones.}
    \label{fig:comp}
\end{figure}

\vspace{1em}
\subsection[Gráficas del error]{Gráficas del error $\boldsymbol{\varepsilon_{(x_i)}}$}
Por la forma de la EDO, los errores al aplicar los tres métodos crece de forma exponencial, tal y como puede apreciarse en las gráficas.

La figura \ref{fig:euler} muestra que el error del método de Euler es mucho más grande que el de los otros dos métodos (siendo las gráficas de estos muy similares). Lo anterior puede ser fácilmente verificado observando la escala.

\begin{figure}[H]
    \centering
    \includegraphics{imgs/plot-euler.pdf}
    \caption{Gráfica del error para el método de Euler.}
    \label{fig:euler}
\end{figure}

\begin{figure}[H]
    \centering
    \includegraphics{imgs/plot-taylor.pdf}
    \caption{Gráfica del error para el método de Taylor de cuarto orden.}
    \label{fig:taylor}
\end{figure}

\begin{figure}[H]
    \centering
    \includegraphics{imgs/plot-runge_kutta.pdf}
    \caption{Gráfica del error para el método de Runge--Kutta.}
    \label{fig:runge_kutta}
\end{figure}

\end{document}
