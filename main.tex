% mklatex 
\documentclass[english,spanish,Ce-table,Ce-theorem]{CabesHW}
\usepackage[utf8]{inputenc}
\usepackage{parskip}
\usepackage{babel}
\usepackage{hyperref}
\usepackage{algpseudocodex}
\usepackage{minted}

%%%%%%%%%%%%%%%%%%%%%%%%%%%%%%%%%
%% Configuraciones adicionales %%
%%%%%%%%%%%%%%%%%%%%%%%%%%%%%%%%%
\makeatletter

%% Algpseudocodex restyling %%
\renewcommand{\algpx@commentFormat}[1]{%
	\ifbool{algpx@italicComments}{%
		\textsl{\textcolor{\algpx@commentColor}{#1}}%
	}{%
		\textcolor{\algpx@commentColor}{#1}%
	}%
}

%% myalg environment %%
\newenvironment{myalg}{%algorithm
    \begin{minipage}{0.8\textwidth}
    \selectlanguage{english}
    \begin{algorithmic}%
}{%
    \end{algorithmic}
    \end{minipage}%
}

%% Float new environments %%
\newfloat{algorithm}{tbp}{loa}
\floatname{algorithm}{Algoritmo}

\makeatother
\endinput
\decimalpoint
\usemintedstyle{lovelace}
\hypersetup{
    colorlinks=true,
    allcolors=blue!65!black
}

\institute{Escuela de Ciencias Físicas y Matemática}
\title{Proyecto 4}
\author{Amado Cabrera, Denilson Mendoza y Mariana Pérez}
\id{201905757, 202005724 y 201901040}
\date{\today}
\course{Laboratorio de simulación}
\professor{Carlos Soto}

\begin{document}

\maketitle

\begin{abstract}
Escribir el \textit{abstract}
\end{abstract}

\section{Marco teórico}
\subsection{Método de Euler}
El método de Euler permite obtener aproximaciones $w_i \approx y_{(x_i)}$ al problema de valor inicial bien planteado
\begin{align*}
\label{1.1}
    \td{y}{t} = f_{(x,y)}, \qquad a \leq t \leq b, \qquad y_{(a)} = \alpha. \tag{1.1}
\end{align*}

Este método no brinda una aproximación continua a la solución $y_{(x)}$, sino que genera aproximaciones de $y_{(x)}$ para varios valores de $x$ en el intervalo $[a, b]$. Dichos valores, denotados como $x_i$ y llamados \textit{puntos de malla}, deben estar distribuidos equitativamente en el intervalo. Para ello, se toma un número entero positivo $N$ (que representa la cantidad de pasos en el intervalo) y se seleccionan los puntos de malla
\[ x_i = a + ih, \qquad \text{para cada $i = 0, 1, \ldots, N$}; \]
donde $h = (b-a)/N = x_{i+1} - x_i$ es la distancia que existe entre cada $x_i$, por lo que $h$ recibe el nombre de \textit{tamaño del paso}.

Se usará el teorema de Taylor para derivar el método de Euler. Suponer que la solución de la ecuación  \eqref{1.1} tiene dos derivadas continuas en el intervalo $[a, b]$, por lo que para cada $i = 0, 1, \ldots, N-1$
\begin{align*}
     y_{(x)} &= y_{x_i} + (x-x_i)y'_{(x_i)} + \frac{(x-x_i)^2}{2!}y''_{(\xi_i)}.
\end{align*} 
Valuando la expresión en $x=x_{i+1}$:
\begin{align*}
     y_{(x_{i+1})} &= y_{x_i} + (x_{i+1}-x_i)y'_{(x_i)} + \frac{(x_{i+1}-x_i)^2}{2!}y''_{(\xi_i)}\\
     &= y_{x_i} + hy'_{(x_i)} + \frac{h^2}{2!}y''_{(\xi_i)}.
\end{align*}
El método de Euler se consigue descartando el término restante que incluye $\xi_i$:
\begin{align*}
    w_0 &= \alpha,\\
    w_{i+1} &= w_i + h f_{(t_i, w_i)}, \qquad \text{para cada $i = 0, 1, \ldots, N-1$}.
\end{align*}


\begin{algorithm}[H]
    \centering
    \begin{myalg}[1]
    \LComment{Método de Euler para resolución numérica de EDO}
    \Function{Edo}{$x$, $y$}
        \State \Output $0.16 y$
    \EndFunction
    \State \phantom{}
    \State $x_i \gets 0$ \Comment{Valor inicial de $x$}
    \State $y_i \gets 367$ \Comment{Valor inicial de $y$}
    \State $x_f \gets 20$ \Comment{Valor final de $x$}
    \State $n \gets 100$ \Comment{Número máximo de pasos}
    \State \phantom{}
    \State $h \gets (x_f - x_i)/n$ \Comment{Valor del paso}
    \For{$i = 0,\ldots,n$}
        \State \Output $(i, x_i, y_i)$
        \State \phantom{}
        \State $y_i \gets y_i + h\cdot\Call{Edo}{x_i, y_i}$
        \State $x_i \gets x_i + h$
    \EndFor
    \end{myalg}
    \caption{Pseudo--código para el método de Euler.}
    \label{alg:euler}
\end{algorithm}

\vspace{2em}
\subsection{Método de Taylor}
Suponer que la solución a la ecuación
\[ \td{y}{x} = f_{(x,y)}, \qquad a \leq t \leq b, \qquad y_{(a)} = \alpha \]
tiene $n+1$ derivadas continuas.

Derivando repetidas veces la solución se obtiene
\begin{align*}
\label{tay2.1}
    y^{(k)}_{(x)} &= f^{(k-1)}_{(x, y_{(x)})}. \tag{2.1}
\end{align*}

Empleando el $n$-ésimo polinomio de Taylor al rededor de $x_i$, expandir la solución $y_{(x)}$ y posteriormente valuarla en $x= x_{i+1}$:
\begin{align*}
    y_{(x)} &= y_{x_i} + (x-x_i)y'_{(x_i)} + \frac{(x-x_i)^2}{2!}y''_{(x_i)} + \ldots + \frac{(x-x_i)^{n}}{n!}y^{(n)}_{(x_i)} + \frac{(t-x_i)^{n+1}}{(n+1)!}y^{(n+1)}_{(\xi_i)}\\
    y_{(x_{i+1})} &= y_{x_i} + (x_{i+1}-x_i)y'_{(x_i)} + \frac{(x_{i+1}-x_i)^2}{2!}y''_{(x_i)} + \ldots + \frac{(x_{i+1}-x_i)^{n}}{n!}y^{(n)}_{(\xi_i)} + \frac{(x_{i+1}-x_i)^{n+1}}{(n+1)!}y^{(n+1)}_{(x_i)}\\
\label{tay2.2}
    y_{(x_{i+1})} &= y_{x_i} + hy'_{(x_i)} + \frac{h^2}{2!}y''_{(x_i)} + \ldots + \frac{h^{n}}{n!}y^{(n)}_{(x_i)} + \frac{h^{n+1}}{(n+1)!}y^{(n+1)}_{(\xi_i)}, \tag{2.2}
\end{align*}
para algún $\xi_i$ en $(x_i, x_{i+1})$.

Reemplazando \eqref{tay2.1} en \eqref{tay2.2}, se consigue
\begin{align*}
    y_{(x_{i+1})} &= y_{x_i} + hf_{(x_i, y_{(x_i)})} + \frac{h^2}{2!}f'_{(x_i, y_{(x_i)})} + \ldots + \frac{h^{n}}{n!}f^{(n-1)}_{(x_i, y_{(x_i)})} + \frac{h^{n+1}}{(n+1)!}f^{(n)}_{(\xi_i, y_{(\xi_i)})}
\end{align*}

El método de Taylor de orden $n$ se consigue descartando el término restante que incluye $\xi_i$:
\begin{align*}
    w_0 &= \alpha\\
    w_{i+1} &= w_i + hT^{(n)}_{(x_i, w_i)}, \qquad \text{para cada $i = 0, 1, \ldots, N-1$}.
\end{align*}
donde
\begin{align*}
    T^{(n)}_{(x_i, w_i)} &= f_{(x_i, y_{(x_i)})} + \frac{h}{2!}f'_{(x_i, y_{(x_i)})} + \ldots + \frac{h^{n-1}}{n!}f^{(n-1)}_{(x_i, y_{(x_i)})} 
\end{align*}


\begin{algorithm}[H]
    \centering
    \begin{myalg}[1]
    \LComment{Método de Taylor para resolución numérica de EDO}
    \Function{Edo}{$x$, $y$} \Comment{Edo y sus derivadas}
        \State \Output $0.16 y$
    \EndFunction
    \State \phantom{}
    \Function{dEdo}{$x$, $y$} \Comment{Derivada 1}
        \State \Output $0.16 \Call{Edo}{x,y}$
    \EndFunction
    \State \phantom{}
    \Function{ddEdo}{$x$, $y$} \Comment{Derivada 2}
        \State \Output $0.16 \Call{dEdo}{x,y}$
    \EndFunction
    \State \phantom{}
    \Function{dddEdo}{$x$, $y$} \Comment{Derivada 3}
        \State \Output $0.16 \Call{ddEdo}{x,y}$
    \EndFunction
    \State \phantom{}
    \State $x_i \gets 0$ \Comment{Valor inicial de $x$}
    \State $y_i \gets 367$ \Comment{Valor inicial de $y$}
    \State $x_f \gets 20$ \Comment{Valor final de $x$}
    \State $n \gets 100$ \Comment{Número máximo de pasos}
    \State \phantom{}
    \State $h \gets (x_f - x_i)/n$ \Comment{Valor del paso}
    \For{$i = 0,\ldots,n$}
        \State \Output $(i, x_i, y_i)$
        \State \phantom{}
        \State $y_i \gets y_i + h(\Call{Edo}{x_i, y_i} + h\cdot\Call{dEdo}{x_i, y_i}/2 + $
        \Statex \hspace{5em}$+ h^2\cdot\Call{ddEdo}{x_i, y_i}/6 + h^3\cdot\Call{dddEdo}{x_i, y_i}/24)$
        \State $x_i \gets x_i + h$
    \EndFor
    \end{myalg}
    \caption{Pseudo--código para el método de Taylor.}
    \label{alg:taylor}
\end{algorithm}

\vspace{2em}
\subsection{Método de Runge-Kutta}
Los métodos de Runge-Kutta tienen el error de truncamiento local de alto orden de los métodos de Taylor pero elimina la necesidad de calcular y evaluar las derivadas de $f(t, y)$. Antes de presentar las ideas detrás de su derivación, necesitamos considerar el Teorema de Taylor en dos variables. La prueba de este resultado se puede encontrar en cualquier libro estándar sobre cálculo avanzado.

\begin{theorem}
Suponga que $f (t, y)$ y todas sus derivadas parciales de orden menor o igual a $n + 1$ son continua en $ D = \{(t, y)\:|\:a \leq t \leq b,\, c \leq y \leq d\}$, y sea $(t_0,y_0)$ $\in D$. Para cada $(t, y) \in D$, existe $\xi$ entre $t$  y $t_0$ y $\mu$ entre  $y$ y $y_0$ con
\end{theorem}

Para derivar el método de Runge-Kutta de cuarto orden, se requiere hallar valores de $a_1$, $\alpha_1$ y $\beta_1$ tales que $f_{(x + \alpha_1, y + \beta_1)}$ aproxime a
\begin{align*}
    T^{(n)}_{(x, y)} = f_{(x, y)} + \frac{h}{2}f'_{(x, y)}
\end{align*}
\begin{align*}
    w_0 &= \alpha\\
    w_{i+1} &= sw_i + \frac{1}{6}(k_1 + 2k_2 + 2k_3 + k_4), \qquad \text{para cada $i= 0, 1, \ldots, N-1;$}
\end{align*}
donde
\begin{align*}
    k_1 &= hf_{(x_i, w_i)}\\
    k_2 &= hf_{(x_i + \frac{h}{2}, w_i + \frac{1}{2}k_1)}\\
    k_3 &= hf_{(x_i + \frac{h}{2}, w_i + \frac{1}{2}k_2)}\\
    k_4 &= hf_{(x_{i+1}, w_i + k_3)}
\end{align*}

\begin{algorithm}[H]
    \centering
    \begin{myalg}[1]
    \LComment{Método de Runge--Kutta para resolución numérica de EDO}
    \Function{Edo}{$x$, $y$}
        \State \Output $0.16 y$
    \EndFunction
    \State \phantom{}
    \State $x_i \gets 0$ \Comment{Valor inicial de $x$}
    \State $y_i \gets 367$ \Comment{Valor inicial de $y$}
    \State $x_f \gets 20$ \Comment{Valor final de $x$}
    \State $n \gets 100$ \Comment{Número máximo de pasos}
    \State \phantom{}
    \State $h \gets (x_f - x_i)/n$ \Comment{Valor del paso}
    \For{$i = 0,\ldots,n$}
        \State \Output $(i, x_i, y_i)$
        \State \phantom{}
        \State $k_1 \gets h\cdot\Call{Edo}{x_i, y_i}$
        \State $k_2 \gets h\cdot\Call{Edo}{x_i+h/2, y_i+k_1/2}$
        \State $k_3 \gets h\cdot\Call{Edo}{x_i+h/2, y_i+k_2/2}$
        \State $k_4 \gets h\cdot\Call{Edo}{x_i+h, y_i+k_3}$
        \State \phantom{}
        \State $y_i \gets y_i + h(k_1 + 2k_2 + 2k_3 + k_4)/6$
        \State $x_i \gets x_i + h$
    \EndFor
    \end{myalg}
    \caption{Pseudo--código para el método de Runge--Kutta.}
    \label{alg:runge-kutta}
\end{algorithm}

\section{Resultados}
% (\$|\\begin\{align\*\})[(.*)t(.*)]+(\$|\\end\{align\*\})

\end{document}
